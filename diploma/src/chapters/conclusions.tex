\chapter{Conclusions}
\label{chapter:conclusions}

In this chapter we conclude our description on this project. In the first section \labelindexref{Summary}{sec:summary} we'll do a summary of how and what we managed to prove in this paper.

In the second section, \labelindexref{Further development}{sec:further-development} we will discuss what could be developed next, based on our example described in the \labelindexref{Implementation}{chapter:implementation}

Finally I will conclude with some personal appreciations in the last section \labelindexref{Final Appreciations}{sec:final-appreciations}

\section{Summary}
\label{sec:summary}

\todo{In principiu cu bullet points}

\section{Further developments}
\label{sec:further-developments}

The purpose of the project was only to demonstrate that it is possible to build a front-end that is agnostic of it's back-end. Thus the relations between the data models we constructed were mostly \textbf{one-to-many} in the case of an \texttt{author} and his related \texttt{posts}. While this helps us to prove our point and have our front-end explore trough the post resource, the author that is assigned to it, we can encounter more complex cases in real life that.

The next step in our implementation would be integrating \textbf{many-to-many} relations between the data entities. A concrete example in our implementation would be having multiple authors assigned to multiple posts and being able to edit the authors assigned to a post accordingly.

\section{Final Appreciations}
\label{sec:final-appreciations}
\todo{De zis ce am invatat in tot timpul in care am lucrat la proiect?}

