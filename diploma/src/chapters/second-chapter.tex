\chapter{State of the art}
\label{chapter:state}

In this chapter we will start to describe the technologies and frameworks used for both back-end and front-end implementations, followed by presenting how they interact as a system in the next chapter~\ref{chapter:implementation}. For serving the web-page we used Django~\ref{sub-sec:django} and for delivering the and the REST API we used a Django compatible extension called Tastypie~\ref{sub-sec:tastypie}, the Django compatible solution. For rendering the front-end components we used Facebook's React.js framework~\ref{sub-sec:react} and also Bootstrap/Flexbox~\ref{sub-sec:bootstrap} for styling.

\section{Back-end Frameworks}
\label{sec:backend}

\subsection{Django}
\label{sub-sec:django}

For our back-end framework we used Django mainly because of it's performance, scalability, stability and the flexibility that it gives to developers. Being a free and open source project, Django benefits of a large community support therefore having an extensive official documentation, a vast collection of tutorials available online and many more extensions available to install trough the built in PIP (Python Package Index) installer.

Django is a web application framework written in Python with a ``batteries-included'' philosophy. The principle behind batteries-included is that the common functionality for building web applications should come with the framework instead of as separate libraries. Therefore Django can provide developers with all the tools needed to put up a web page. It has a built in relational database for storing information, a templating system for rendering pages and a URL dispatcher for resolving page requests. The framework also comes with more advanced features like XSS (Cross site scripting), CSRF (Cross site request forgery) and SQL Injection protection but also an authentication and admin panel modules.

Django's architecture is described by the developers\footnote{https://docs.djangoproject.com/en/1.8/faq/general/\#django-appears-to-be-a-mvc-framework-but-you-call-the-controller-the-view-and-the-view-the-template-how-come-you-don-t-use-the-standard-names\label{note1}} as a MTV, which stands for ``Model-Template-View,, even if it closely resembles a classic MVC (Model View Controller) pattern which many may be familiar with. A high level view of Django's architecture can be seen in \labelindexref{Figure}{img:django-arch}.

\fig[scale=0.5]{src/img/Django-Arch.png}{img:django-arch}{Django architecture \cite{django-arch} }

\subsubsection{Model}
\label{sub-sub-sec:model}

The "Model" consists of an object-relational mapper that translates data models defined in Python classes to classic relational database tables. Django's default relational database management system is SQLite.

Models are represented by Python classes and contain essential fields and behaviors of the stored data and represent the single source of information about that data. Every class inherits the \textbf{django.db.models.Model} and each model attribute represents a database field. An very basic example can be seen in \labelindexref{Listing}{lst:model-py}.

\lstinputlisting[label={lst:model-py},caption=Post model class,language=Python]{src/code/python/modelexample.py}

We can see that we have two character type fields and one date type field that map to the corresponding columns in \labelindexref{Listing}{lst:model-sql}.

\lstinputlisting[label={lst:model-sql},caption=Post SQL table,language=SQL]{src/code/python/examplesql.sql}

\subsubsection{Template}
\label{sub-sub-sec:template}

The "Template" consists of a web templating system that handles the page rendering on the client part. This is also why it is called the "Template" even if it basically resembles the MVC "View" component. A Django template is composed of the static parts of the desired HTML output as well as special syntax that describes describing how dynamic content will be inserted. An example of rendering a poll question with it's available choices can be seen in \labelindexref{Listing}{lst:template-example}.

\lstinputlisting[label={lst:template-example},caption=Using Django templates,language=html]{src/code/html/template-example.html}

We can see that the Django that trough \texttt{\{\{var\}\}} specific variables are inserted into the HTML that will be generated, like the question text or the choices text. Also, Django provides specific built-in template tags like \texttt{\{\% for in \%\}} for looping over arrays, handling conditional statements, filtering etc.

\subsubsection{View}
\label{sub-sub-sec:view}

The "View" is a regular-expression-based URL dispatcher that maps the URL to a view function and calls it. The view function can also check if a cached version of the page is available and skip the following steps. The component was named "View" by the developers because the callback function of the dispatcher describes which data is presented to de user. The "View" can be seen as an MVC "Controller" because the view functions performs the requested action which usually implies reading or writing to the database. An example of a urls.py file can be seen in \labelindexref{Listing}{lst:view-example}.

\lstinputlisting[label={lst:view-example},caption=Urls.py file,language=Python]{src/code/python/view-example.py}

We can see that each URL pattern is matched via the regular expression set as the first parameter of the \texttt{url()} function (eg. \texttt{r'\^{}\$'}) and calls the function callback (the second parameter) with following optional parameters. In the first case the regular-expression will match the empty string with the index function from the views module.



\subsection{Tastypie}
\label{sub-sec:tastypie}

Tastypie\footnote{\url{http://tastypieapi.org/}} is a webservice API framework for Django. It provides a convenient, yet powerful and highly customizable abstraction for creating REST-style interfaces. Like Django, Tastypie is an open source, community backed project. Some of it's main features are:

\begin{itemize}
	\item Full GET/POST/PUT/DELETE/PATCH support
	\item Designed to be extended at every turn
	\item Includes a variety of serialization formats (JSON/XML/YAML/bplist)
	\item HATEOAS by default
	\item Well-tested and well-documented
\end{itemize}

Tastypie's basic functionality is taking data represented in the Django models (described by python classes), serializing it and sending the resulted data to the client that consumes the API. During the request/response cycle, Tastypie uses a large portion of the standard Django bahaviour and adds on top of that a ,,hydrate/dehydrate'' cycle. An example for requesting data from a sample endpoint like \texttt{api/v1/author/?format=json} can be summarized as follows:

\begin{itemize}
	\item The Django URL dispatcher checks the requested URL in the \texttt{Resource.urls} and on a match for the list view calls \texttt{Resource.dispatch}
	\item The \texttt{dispatch} method checks if the request is valid by establishing if the used HTTP method is valid, if the requesting user is authenticated and authorized and if the resource has a handle  request method. If all checks are passed the \texttt{dispatch} calls \texttt{get\_list}.
	\item The \texttt{get\_list} method fetches the available ModelResource objects, sorts them and applies \texttt{full\_dehydrate} on each one of them.
	\item The \texttt{full\_dehydrate} has the purpose of taking the data from the ModelResource objects and transforming it into a much simpler Tastypie specific data structure, usually a dictionary of simple data types and. Therefore the method applies each object's  \texttt{dehydrate} method to get a simpler data structure for serializing. The process is similar on  POST/PUT/PATCH requests where the \texttt{hydrate} method is applied instead, hence the ,,hydrate/dehydrate'' cycle.
	\item In the end, the \texttt{create\_response} method serializes the data given to it in the requested format (XML, JSON) and returns HttpResponse (200 OK) with the resulted data.
\end{itemize}


\section{Frontend-end Frameworks}
\label{sec:frontend}

\subsection{React.js}
\label{sub-sec:react}

\todo{Describe the React framework}

\subsection{Bootstrap/Flexbox}
\label{sub-sec:bootstrap}

For styling our HTML components we used the Bootstrap framework to achieve an experience similar to a real product. For laying out and aligning the panels that are auto-generated by exploring the Tastypie resource, we used the Flexbox module. 

\subsubsection{Bootstrap}
\label{sub-sub-sec:bootstrap}

Bootstrap is a powerful front-end framework created at Twitter by Mark Otto and Jacob Thornton\footnote{\url{https://github.com/twbs/bootstrap}}, that features a rich collection of tools for faster development and quality work. The most appreciated features are the custom HTML and CSS components, plenty javascript plugins and a popular responsive page layout system. The project is also an open-source community backed project and it's main goal is to ease web development by giving the developers the ability to have consistent styling with minimum effort.

With Bootstrap we can simply structure the DOM with components like Panels by setting custom classes on the elements. An example can be seen in \labelindexref{Listing}{lst:bootstrap} and the resulted component in \labelindexref{Figure}{img:bootstrap-panel}.

\lstinputlisting[label={lst:bootstrap},caption=Panel with title ,language=Html]{src/code/html/panel.html}

\fig[scale=0.5]{src/img/bootstrap-panel.png}{img:bootstrap-panel}{Panel with title}


\subsubsection{Flexbox}
\label{sub-sub-sec:flexbox}

The Flexbox Layout (Flexible Box) module was built in the CSS3 standard, with the main goal of trying to properly alter width/height of items inside of a container, even if their sizes were unknown, hence the "flexible box" naming.

In the flex layout model, the children of a flex container can be laid out in any direction, and can “flex” their sizes, either growing to fill unused space or shrinking to avoid overflowing the parent\cite{flexbox}. An overview of the flex container can be seen in \labelindexref{Figure}{img:flexbox}.

\fig[scale=0.5]{src/img/flexbox.png}{img:flexbox}{Flex container\cite{flexbox} }

We can see the main flex container being populated by flex items. In the flex layout the items will be laid out following either the \texttt{main axis} (from \texttt{main-start} to \texttt{main-end}) or the \texttt{cross axis} (from \texttt{cross-start} to \texttt{cross-end}).

The container has a collection of properties for laying out the items like: \texttt{flex-direction, flex-wrap, flex-flow, align-content, align-items, justify-content} which can be used for item layout and alignment. An example using some of theese can be seen in \labelindexref{Figure}{img:flexbox2} presented by the W3C specification \cite{flexbox}.

\fig[scale=0.5]{src/img/flexbox2.png}{img:flexbox2}{Flexbox usage example \cite{flexbox} }

The \texttt{flex-flow} property is a shorthand for \texttt{flex-direction} and \texttt{flex-wrap} properties. It first establishes the main axis and direction of the layout (\texttt{row-reverse}), and sets the items to wrap if they don't fit on the same row in the upwards direction (\texttt{upwards}).