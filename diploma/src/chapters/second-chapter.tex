\chapter{State of the art}
\label{chapter:state}

In this chapter we will start to describe the technologies and frameworks used for both back-end and front-end implementations, followed by presenting them as a whole in the next chapter \ref{chapter:implementation}. For serving the web-page we used Django \ref{sub-sec:django} and for delivering the and the REST API we used Tastypie \ref{sub-sec:tastypie}, the Django compatible solution. For the logic and for rendering the front-end components we used Facebook's React.js framework \ref{sub-sec:react} and also Bootstrap/Flexbox \ref{sub-sec:bootstrap} for styling.

\section{Back-end Frameworks}
\label{sec:backend}

\subsection{Django}
\label{sub-sec:django}

For our back-end framework we used Django mainly because of it's performance, stability and the flexibility that it gives to developers. Also being a free and open source project, Django benefits of a large community support therefore having a very detailed official documentation and a vast collection of tutorials available online.

Django is a web application framework written in Python with a "batteries-included" philosophy. The principle behind batteries-included is that the common functionality for building web applications should come with the framework instead of as separate libraries.

Even if Django's architecture resembles an MVC (Model View Controller), its developers call it a MTV (Model Template View) framework\footnote{https://docs.djangoproject.com/en/1.8/faq/general/\#django-appears-to-be-a-mvc-framework-but-you-call-the-controller-the-view-and-the-view-the-template-how-come-you-don-t-use-the-standard-names\label{note1}}. A high level view of Django's architecture can be seen in
\begin{itemize}
	\item The "Model" consists of an object-relational mapper that translates data models defined in Python classes to classic relational database tables.
	\item The "View" consists of a web templating system that handles the page rendering on the client part. This is also why it is called the "Template" instead of "View".
	\item The "Controller" is a regular-expression-based URL dispatcher that handles the page requests. It is seen as a "View" by the developers because the callback function of the dispatcher describes which data is presented to de user.
\end{itemize}

\subsection{Tastypie}
\label{sub-sec:tastypie}

\subsubsection{API}


\subsubsection{Schema}



\section{Frontend-end Frameworks}
\label{sec:frontend}

\subsection{Bootstrap}
\label{sub-sec:bootstrap}


\subsection{React.js}
\label{sub-sec:react}

