\chapter{Implementation}
\label{chapter:implementation}

In this chapter we present the main implementation idea and how all the technologies work together, starting with the \labelindexref{Overview}{sec:overview} section.

Secondly we describe the Django data models and how they \todo{CE mai adaug aici?}

Then we present how Tastypie uses the Django data models to generate the API resource with its corresponding schema in section \labelindexref{REST API}{sec:api}.

Finally in the \labelindexref{React Components}{sec:react0components} section, we present how these work together to explore the resource from the given API URI and how they auto-generate UI components based on the received JSON data.

\section{Overview}
\label{sec:overview}

\todo{Schema cu tehnologiile impreuna?}

\section{Django Models}
\label{sec:django-models}

\todo{cod modele User + Post}
%eventual translatarea in SQL ? desi cam useless

\section{REST API}
\label{sec:api}

\todo{Cod django pt resurse}
\todo{cod de output API}
\todo{cod de output schema}

\section{React Components}
\label{sec:react0components}

\todo{poza smechera cu structura componentelor}
\todo{schema cu ce face FormBox ? flow-ul general al datelor}
\todo{descris GenericForm}
\todo{Descris componentele separate}