\chapter{Implementation}
\label{chapter:implementation}

In this chapter we present the main implementation idea and how all the technologies work together, starting with the \labelindexref{Overview}{sec:overview} section.

Secondly we describe the Django data models and how they \todo{CE mai adaug aici?}

Then we present how Tastypie uses the Django data models to generate the API resource with its corresponding schema in section \labelindexref{REST API}{sec:api}.

Finally in the \labelindexref{React Components}{sec:react0components} section, we present how these work together to explore the resource from the given API URI and how they auto-generate UI components based on the received JSON data.

\section{Overview}
\label{sec:overview}

The web application's goal is to demonstrate that an API can be explored and UI components can be auto-generated from the data that it receives. Therefore, for this to work, we established a Django based back-end that holds the requested or modified data and we set up a REST API using Tastypie. Next we implemented our front-end in React to explore the data received from the API and generate custom components. A general overview of all the application components and how they interact is presented in \labelindexref{Figure}{img:schema-generala}. %\lstinputlisting[label={lst:flex-flow},caption=Flexbox flex-flow usage ,language=html]{src/code/html/flex-flow.html}

\fig[scale=0.45]{src/img/schema-generala.png}{img:schema-generala}{General application architecture}

\begin{enumerate}
	\item \texttt{Register} - First the Django models classes are are defined (\labelindexref{Section}{sec:django-models}) and the SQL database tables are created and populated. After that a Django Tastypie Resource is declared using the previous defined models (\labelindexref{Section}{sec:api}) and the \texttt{Register} function is called using this resource to create the API.
	\item \texttt{Initial GET} - The UI logic uses two AJAX calls (HTTP GETs) to retrieve data and schema about specified endpoint URI, with the purpose of rendering generic components.
	\item \texttt{Response} - The Tastypie API queries the database from the Django server and returns responses to the previous two requests. The data returned is structured as a JSON and it describes the objects that are stored at that specific endpoint. Also along with the data the APU returns the JSON Schema that describes the fields of the resource.
	\item \texttt{GET/POST/PATCH/DELETE} - Finally the UI renders the data received from the initial GET and. Now the user can trigger browser events like deleting specific panels, adding new data or exploring and modifying existing data, thus sending the corresponding HTTP requests to the API for the updates.
	\item \texttt{Update DB} - Once the API recieves new requests from the UI it queries the database and updates it correspondingly trough the hydrate/dehydrate cycle.
	
\end{enumerate}

The next sections get into more detail on each of the steps described above.

\section{Django Data Models}
\label{sec:django-models}

For the purpose of the project we only needed a functional REST API and a way to render the main HTML page. The first step of establishing the REST API was to define the Django data models. For testing purposes we defined a \texttt{Post} entity, similar to a usual blog post, with the following properties:

\begin{itemize}
	\item \texttt{title} - The title of the post.
	\item \texttt{author} - A related entity that describes the data about the author of the specific post.
	\item \texttt{created\_at} - A field that describes the date and time the post was created at.
	\item \texttt{content} - The actual text of the post.
\end{itemize}

The \texttt{author} field is important for the purpose of demonstrating the capability to explore related fields and describe them. The \texttt{author} data model is actually inherited from Django's \texttt{contrib.auth.models User} model which is composed of standard fields as: \texttt{username, email, first name, last name, date joined} and also fields that we chose to exclude trough filters like: \texttt{password, is\_active, is\_staff, is\_superuser}.

In \labelindexref{Listing}{lst:post-model} we present the Django Post model class implementation. As all Django model classes, it inherits the \texttt{models.Model} class and its standard field types.

\lstinputlisting[label={lst:post-model},caption=Post model class,language=Python]{src/code/python/post-model.py}

Field types are classes that describe and encapsulate certain types of data. As it is presented in the listing above field types can hold additional parameters as \texttt{help\_text}, which is used to provide additional information about the field and subsequently displayed in the REST API's JSON Schema. Another optional parameter is \texttt{DateTimeField}'s \texttt{auto\_now\_add} which is set to \texttt{True} to automatically set the field to now when the object is first created.
		
%eventual translatarea in SQL ? desi cam useless

\section{REST API}
\label{sec:api}

The REST API set up trough Django Tastypie is the most important part of our back-end. It takes the data models described in the previous section and outputs them in a RESTful way. We will next present how we set up the API and also its resulted output.

\subsection{Resources}
\label{sub-sec:resources}

Previously we described the data models for a generic blog post and its related author model. Based on those models we created API resources that are able to expose that data.

\subsubsection{Post Resource}
\label{sub-sub-sec:post-resource}

In \labelindexref{Listing}{lst:cod-resursa-post} we present the implementation for the \texttt{PostResource} resource. The resource inherits the base methods from the \texttt{ModelResource} class that itself is a subclass of \texttt{Resource}, designed to work with Django’s \texttt{Models}. 

\lstinputlisting[label={lst:cod-resursa-post},caption=Post resource,language=Python]{src/code/python/cod-resursa-post.py}	

The inner \texttt{Meta} allows for class-level configuration of how the Resource should behave. Therefore we can set parameters like \texttt{resource\_name}, \texttt{authorization} and the most important \texttt{queryset}.The \texttt{queryset} provides the resource with the set of Django models to respond with. Also because the \texttt{author} field is a related data model we want to provide it with a \texttt{ForeignKey} to the \texttt{UserResource} to provide a link to the related URI. A more detail output of the API will be presented in

In case of a related component the JSON Schema should describe the URI of the parent resource. For this, as seen in lines 11-14 in the above listing,  we overwrote the \texttt{build\_schema} method in the \texttt{Post} resource class to add another field called \texttt{resource} that would expose the \texttt{author} resource URI.


\subsubsection{User Resource}
\label{sub-sub-sec:user-resource}

The \texttt{UserResource} follows the same design that we presented in the section before, on the \texttt{PostResource}. While it will not have a modified \texttt{build\_schema} method it will have some additional \texttt{Meta} fields. \labelindexref{Listing}{lst:cod-resursa-user} shows only what changed in the \texttt{Meta} class. For security reasons we excluded the fields like \texttt{password, is\_active, is\_staff, is\_superuser} trough the \texttt{excludes} option. The \texttt{filtering} option provides the resource with a list of fields that will accept client filtering on. In this case by setting the `\texttt{'username': ALL} option we can now use specific queries like \texttt{api/v1/author/?username=flaviusone\&format=json}.

\lstinputlisting[label={lst:cod-resursa-user},caption=User resource,language=Python]{src/code/python/cod-resursa-user.py}

After setting up the resources we continued by setting up the API for the URL dispatcher in Django's \texttt{urls.py} file. Next, we instantiated the resources and we called the Tastypie provided constructor \texttt{Api(api\_name='v1')} with a parameter for setting an API name. In the end we called the \texttt{register()} method on the API object with both resources as parameters and we added the \texttt{url()} call in the \texttt{urlpatterns} array as follows \texttt{url(r'\^{}api/', include(v1\_api.urls))}

\subsection{Data Output}
\label{sub-sec:output}

With the API properly setup we now have output on the resource specific URI, in our case \texttt{/posts/api/v1/post/?format=json}. In \labelindexref{Listing}{lst:data-resursa-post} we present a sample output of the \texttt{Post} resource (only one object displayed).

\lstinputlisting[label={lst:data-resursa-post},caption= Post resource data output,language=Java]{src/code/js/json-data-post.js}	

The listing above shows the return of a single object with two proprieties: \texttt{meta} and \texttt{objects}. The \texttt{meta} object provides info about the number of objects displayed by in the response. Some important parameters here are \texttt{limit} and \texttt{next} which describe how many objects can be returned in one response and what is the path for the next page of objects in case the response is too big and the system automatically splits the response into separate calls to reduce bandwidth. 

The \texttt{objects} property describes an array of objects that are located at the specified resource. The data is also described in JSON format, but it can also be formatted as XML by using the \texttt{?format=xml} in the request call. In the case of the \texttt{author} field the API specifies the related path to the location of the data by obeying HATEOAS principles. By further exploring this URI we get the following additional JSON object, presented in \labelindexref{Listing}{lst:json-author-1}, that further describes the author object. 


\lstinputlisting[label={lst:json-author-1},caption= Author entry,language=Java]{src/code/js/json-author-1.js}


Besides the data output of the \texttt{Post} resource we can also access the JSON Schema available at  \texttt{/posts/api/v1/post/schema/?format=json}. The output of this call can be found in \labelindexref{Appendix}{lst:schema-resursa-post}. The JSON Schema describes the \texttt{type} of each field that is used by the front-end to generate UI components. The standard types described by the Schema can be \texttt{string, integer, datetime} and \texttt{related}. In the case of a \texttt{related} type object the front-end logic will explore the additional \texttt{resource} parameter and display the available data. Another important field used by the front-end logic is the \texttt{readonly} attribute, that points if the object field is editable.	

\section{React Components}
\label{sec:react-components}

As we presented in the \labelindexref{Overview}{sec:overview} section, the main element of the project is the reactive UI. The interface is built to be agnostic of any changes in the back-end by generating specific components based on the data provided and the JSON Schema. In this section we will present how the React components are structured and how they work together.

\subsection{Component Structure}
\label{sub-sec:component-structure}

By its nature, a web application built in React is composed of multiple reusable \texttt{React Components}. In \labelindexref{Figure}{img:generic-form} we present a simplified version of the UI component structure that describes the following elements:

\fig[scale=0.45]{src/img/components.png}{img:components}{React components structure}

\begin{itemize}
	\item \texttt{<FormBox/>} - This is the main component that encapsulates all other. This component also has the main purpose of calling the AJAX requests that are called by other components that are its children.
	\item \texttt{<Navbar/>} - This is not in fact a specific React component but its a structure composed of all the elements that form the page's navigation bar that are actually react components imported from the \texttt{react-bootstrap} package. Some of the components used include:
	  \begin{itemize}
	  	\item \texttt{<Navbar/>} and \texttt{<Nav/>} components, used for encapsulating the URL input form.
	  	\item \texttt{<Input/>} component that describes a simple input form for the URL options.
	  	\item \texttt{<ButtonInput/>} component that describes the submit button for the input form described above
	  \end{itemize}
	\item \texttt{<FormList/>} - This component receives an array of objects from its parent and generates and encapsulates the \texttt{<GenericForm/>} components.
	\item \texttt{<GenericForm/>} - This represents the main component that the user interacts with. It also encapsulates components for each type of object. We will continue to present this component in further detail in  \labelindexref{Section}{sub-sec:generic-form}.
\end{itemize}


\subsection{Data Flow}
\label{sub-sec:data-flow}


\fig[scale=0.45]{src/img/data-flow.png}{img:data-flow}{Data flow}


\subsection{Generic Form}
\label{sub-sec:generic-form}

\todo{Descris GenericForm}
\fig[scale=0.35]{src/img/genericform.png}{img:generic-form}{GenericForm component}
\todo{Descris componentele separate}