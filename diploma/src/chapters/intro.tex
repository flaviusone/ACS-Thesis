\chapter{Introduction}
\label{chapter:intro}

\section{Objectives}
\label{sec:objectives}

The Dynamic Webform Builder project started out as an idea at Hootsuite Romania and had the main goals of:
\begin{itemize}
	
	\item Defining and maintaining a REST (Representational State Transfer) API (Application programming interface) and model schema in a single place (server-side)
	
	\item Auto-exploring the resource and auto-generating generic forms
	
	\item Creating add/edit forms with minimal client-side work (scalable user interface)
	
	\item Have consistent look and experience in all forms
	
\end{itemize}

\section{Motivation}
\label{sec:motivation}

Today's web applications are split into the front-end, usually available on the client side and the back-end often described as the server-side. This division forces developers in big companies to work in specialized teams for the front -end and the back-end development. Ideally developers wish that the front-end and the back-end of the application are agnostic of each other, therefore excluding further problems such as changing a resource URL(Uniform Resource Locator) on the server side and having to manually change it everywhere in the front-end. Avoiding these kind of conflicts will speed up the development process and therefore generating more profit for the company.

For a front-end application to be agnostic of it's back-end it should be able to auto explore a given API endpoint that may change over time. Also the front-end should be able to auto-generate UI components regardless of the data type and structure retrieved from the server.

Therefore this project aims to demonstrate that it is possible to have auto-exploration and auto-generation in a front-end application that is independent of the data changes from the back-end server.


\section{Background}
\label{sec:background}

In this section we will describe the base concepts that are used across this thesis. Therefore we will start by describing the concept of a Web service in section~\ref{sub-sec:web-service} followed by explaining the base concepts behind the REST software architecture in~\ref{sub-sec:rest} and it's HATEOAS constraint described in~\ref{sub-sec:hateoas}.

\subsection{Web service}
\label{sub-sec:web-service}

As the W3C (World Wide Web Consortium) states \cite{W3C}, ``A Web service is a software system designed to support interoperable machine-to-machine interaction over a network. It has an interface described in a machine-processable format (specifically WSDL). Other systems interact with the Web service in a manner prescribed by its description using SOAP-messages, typically conveyed using HTTP with an XML serialization in conjunction with other Web-related standards.'' Therefore we can define a web service in more simpler terms as a framework for a conversation between two computers that are communicating over the web, where a client sends a request message server receives that request, processes it and returns a response message.

Interaction with web services is done typically trough HTTP (Hypertext Transfer Protocol). An example of web service interaction in today's web pages is described by AJAX(Asynchronous JavaScript and XML) requests.   To prevent a full page transition a the browser may initiate an asynchronous request (AJAX call) asking for specific data to be rendered later(eg. rendering an email into the Gmail application). The server will send a response in the form of HTML (HyperText Markup Language) code or just pure data encoded in XML (EXtensible Markup Language) or JSON (JavaScript Object Notation).

\subsubsection{Web API}
\label{sub-sub-sec:web-api}

Even if the messages sent and received from the web service are transmitted trough HTTP, as a developer you still need to have knowledge of the messages format that are being sent or received. Therefore you will need to know details about the service's API.

An Application Programming Interface (API) is a particular
set of rules and specifications that a software program can follow to access and make use of the services
and resources provided by another particular software program that implements that API.

A Web API defines both the server-side API as well as the client browser API and must impose specific elements like the message format or request syntax. Some web services may use SOAP, XML or JSON for data formatting on the request and response calls. On the client side a developer should know if the request syntax may imply using certain URI's (Uniform Resource Identifiers) and specific parameters and data types. On the server side the web service will act on different types of requests that can be associated with HTTP verbs like GET, POST, PUT, PATCH or DELETE. Again, the developer should know the format of the messages received and structure of the data including specific fields and data types.

According to the W3C \cite{W3C} we can identify two major classes:
\begin{itemize}
	\item REST-compliant Web services, in which the primary purpose of the service is to manipulate representations of Web resources using a uniform set of stateless operations.
	\item Arbitrary Web services, in which the service may expose an arbitrary set of operations. 
\end{itemize}

Today web API's have been moving\footnote{\url{http://www.infoq.com/articles/rest-soap}} to a much simpler and well defined representational state architecture and are now called RESTful API's. These systems usually communicate over HTTP with the same HTTP verbs (GET, POST, PUT, PATCH, DELETE) used by browsers to render web pages.

\subsection{REST}
\label{sub-sec:rest}

Hardcore theory TODO
\texttt{}
\subsection{HATEOAS}
\label{sub-sec:hateoas}

Hardcore theory TODO