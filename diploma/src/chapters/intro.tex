\chapter{Introduction}
\label{chapter:intro}

\section{Motivation}
\label{sec:motivation}


\section{Background}
\label{sec:background}

In this section we will describe the base concepts that are used across this thesis. Therefore we will start by describing the concept of a web service in section \ref{sub-sec:web-service} followed by explaining the base concepts behind the REST software architecture in \ref{sub-sec:rest} and it's HATEOAS constraint described in \ref{sub-sec:hateoas}.

\subsection{Web service}
\label{sub-sec:web-service}

According to the W3C, "A Web service is a software system designed to support interoperable machine-to-machine interaction over a network. It has an interface described in a machine-processable format (specifically WSDL). Other systems interact with the Web service in a manner prescribed by its description using SOAP-messages, typically conveyed using HTTP with an XML serialization in conjunction with other Web-related standards."\footnote{\url{http://www.w3.org/TR/2004/NOTE-ws-gloss-20040211/}} Therefore we can define a web service in more simpler terms as a framework for a conversation between two computers that are communicating over the web, where a client sends a request message server receives that request, processes it and returns a response message.


\subsection{REST}
\label{sub-sec:rest}



\subsection{HATEOAS}
\label{sub-sec:hateoas}



\section{Objective}
\label{sec:objective}

